% The document class supplies options to control rendering of some standard
% features in the result.  The goal is for uniform style, so some attention 
% to detail is *vital* with all fields.  Each field (i.e., text inside the
% curly braces below, so the MEng text inside {MEng} for instance) should 
% take into account the following:
%
% - author name       should be formatted as "FirstName LastName"
%   (not "Initial LastName" for example),
% - supervisor name   should be formatted as "Title FirstName LastName"
%   (where Title is "Dr." or "Prof." for example),
% - degree programme  should be "BSc", "MEng", "MSci", "MSc" or "PhD",
% - dissertation title should be correctly capitalised (plus you can have
%   an optional sub-title if appropriate, or leave this field blank),
% - dissertation type should be formatted as one of the following:
%   * for the MEng degree programme either "enterprise" or "research" to
%     reflect the stream,
%   * for the MSc  degree programme "$X/Y/Z$" for a project deemed to be
%     X%, Y% and Z% of type I, II and III.
% - year              should be formatted as a 4-digit year of submission
%   (so 2014 rather than the academic year, say 2013/14 say).

\documentclass[ oneside,% the name of the author
                    author={Michael Wray},
                % the degree programme: BSc, MEng, MSci or MSc. CHANGE AS APPROPRIATE
                    degree={BSc},
                % the dissertation    title (which cannot be blank)
                     title={Some Structural Guidelines for CS Project Dissertations \\ With a Second Line Added to the Title},
                % the unit you are a part of. CHANGE AS APPROPRIATE
                    unit={COMS30045},
                % the dissertation subtitle (which can    be blank)
                    subtitle={And Even A Fancy Subtitle}]{dissertation}

\begin{document}

% =============================================================================

% This section simply introduces the structural guidelines.  It can clearly
% be deleted (or commented out) if you use the file as a template for your
% own dissertation: everything following it is in the correct order to use 
% as is.


% =============================================================================

% This macro creates the standard UoB title page by using information drawn
% from the document class (meaning it is vital you select the correct degree 
% title and so on).

\maketitle

% After the title page (which is a special case in that it is not numbered)
% comes the front matter or preliminaries; this macro signals the start of
% such content, meaning the pages are numbered with Roman numerals.

\frontmatter


%\lstlistoflistings

% The following sections are part of the front matter, but are not generated
% automatically by LaTeX; the use of \chapter* means they are not numbered.

% -----------------------------------------------------------------------------

\chapter*{Abstract}


% -----------------------------------------------------------------------------


\chapter*{Dedication and Acknowledgements}

\vspace{1cm} 

\noindent


% -----------------------------------------------------------------------------

% This macro creates the standard UoB declaration; on the printed hard-copy,
% this must be physically signed by the author in the space indicated.

\makedecl


% -----------------------------------------------------------------------------

% This macro creates the AI declaration; on the printed hard-copy,
% this must be physically signed by the author in the space indicated.

\makeaidecl



% -----------------------------------------------------------------------------

% LaTeX automatically generates a table of contents, plus associated lists 
% of figures and tables.  These are all compulsory parts of the dissertation.

\tableofcontents
\listoffigures
\listoftables

% -----------------------------------------------------------------------------



\chapter*{Ethics Statement}

    \begin{itemize}
        \item This project did not require ethical review, as determined by my supervisor
        \item This project fits within the scope of ethics application 0026, as reviewed by my supervisor
        \item An ethics application for this project was reviewed and approved by the faculty research ethics committee as application.
    \end{itemize}
    
% -----------------------------------------------------------------------------

\chapter*{Summary of Changes}


% -----------------------------------------------------------------------------

\chapter*{Supporting Technologies}



% -----------------------------------------------------------------------------

\chapter*{Notation and Acronyms}




% =============================================================================

% After the front matter comes a number of chapters; under each chapter,
% sections, subsections and even subsubsections are permissible.  The
% pages in this part are numbered with Arabic numerals.  Note that:
%
% - A reference point can be marked using \label{XXX}, and then later
%   referred to via \ref{XXX}; for example Chapter\ref{chap:context}.
% - The chapters are presented here in one file; this can become hard
%   to manage.  An alternative is to save the content in seprate files
%   the use \input{XXX} to import it, which acts like the #include
%   directive in C.

\mainmatter


\chapter{Introduction}
\label{chap:context}


% -----------------------------------------------------------------------------

\chapter{Background}
\label{chap:technical}




% -----------------------------------------------------------------------------

\chapter{Project Execution}
\label{chap:execution}


% -----------------------------------------------------------------------------

\chapter{Critical Evaluation}
\label{chap:evaluation}


% -----------------------------------------------------------------------------

\chapter{Conclusion}
\label{chap:conclusion}


% =============================================================================

% Finally, after the main matter, the back matter is specified.  This is
% typically populated with just the bibliography.  LaTeX deals with these
% in one of two ways, namely
%
% - inline, which roughly means the author specifies entries using the 
%   \bibitem macro and typesets them manually, or
% - using BiBTeX, which means entries are contained in a separate file
%   (which is essentially a databased) then inported; this is the 
%   approach used below, with the databased being dissertation.bib.
%
% Either way, the each entry has a key (or identifier) which can be used
% in the main matter to cite it, e.g., \cite{X}, \cite[Chapter 2}{Y}.
%
% We would recommend using BiBTeX, since it guarantees a consistent referencing style 
% and since many sites (such as dblp) provide references in BiBTeX format. 
% However, note that by default, BiBTeX will ignore capital letters in article titles 
% to ensure consistency of style. This can lead to e.g. "NP-completeness" becoming
% "np-completeness". To avoid this, make sure any capital letters you want to preserve
% are enclosed in braces in the .bib, e.g. "{NP}-completeness".

\backmatter

\bibliography{dissertation}

% -----------------------------------------------------------------------------

% The dissertation concludes with a set of (optional) appendicies; these are 
% the same as chapters in a sense, but once signaled as being appendicies via
% the associated macro, LaTeX manages them appropriatly.

\appendix

\chapter{Appendix A: AI Prompts}
\label{appx:ai_prompt}


% =============================================================================

\end{document}
