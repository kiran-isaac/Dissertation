\chapter{An Additional Derivation Using the Type Checking Algorithm}
\label{appx:example_derive}

\begin{figure}[!h]
\begin{mathpar}
\Gamma = \alpha,\beta,x : \alpha,y : \beta\\
\Delta = \alpha\\

\Infer{\Sub[6]}
  {\Infer{\Var[8]}
      {(x : \alpha) \in \{\Gamma\}}
      {\synjudg{\Gamma}{x}{\alpha}{\Gamma}}
    \\
  \Infer{\SubVar[9]}
      { }
      {\subjudg{\Gamma[\alpha]}{\alpha}{\alpha}{\Gamma[\alpha]}}
  }
  {\chkjudg{\Gamma}{x}{\alpha}{\Gamma}}
\\

\Infer{\Sub[7]}
  {\Infer{\Var[10]}
      {(y : \beta) \in \{\Gamma\}}
      {\synjudg{\Gamma}{y}{\beta}{\Gamma}}
    \\
  \Infer{\SubVar[11]}
      { }
      {\subjudg{\Gamma[\beta]}{\beta}{\beta}{\Gamma[\beta]}}
  }
  {\chkjudg{\Gamma}{y}{\beta}{\Gamma}}

 \Infer{\AllIntro[1]}
   {\Infer{\AllIntro[2]}
   {
    \Infer{\!\ArrIntro[3]}
      {
       \Infer{\!\ArrIntro[4]}
          {
          \Infer{\MyTCRule{\Paircheckrulename}[5]}
            {[6]\chkjudg{\Gamma}{x}{\alpha}{\Gamma} \\ [7]\chkjudg{\Gamma}{y}{\beta}{\Gamma}}
            {\chkjudg{\Gamma}{(x, y)}{(\alpha, \beta)}{\Gamma}}
          }
          {\chkjudg{\alpha,\beta,x : \alpha}{\lam{y} (x, y)}{\beta \fto (\alpha, \beta)}{\Gamma}}
          }
      {{\chkjudg{\alpha,\beta}{\lam{x\; y} (x, y)}{\alpha \fto \beta \fto (\alpha, \beta)}{\Delta, \beta, \{x : \alpha, y : \beta\}}} \\ (= \Gamma)}
   }
   {{\chkjudg{\alpha}{\lam{x\;y} (x, y)}{\alltype{\beta}{\alpha \fto \beta \fto (\alpha, \beta)}}{\{.\}, \alpha, \{.\}}} \\ (= \Delta)}
   }
   {\chkjudg{.}{\lam{x\;y} (x, y)}{\alltype{\alpha\;\beta}{\alpha \fto \beta \fto (\alpha, \beta)}}{.}}

\end{mathpar}
\caption{An example derivation showing how we can typecheck the function $\lam{x\;y} (x, y)$ against $\alltype{\alpha\;\beta}{\alpha \fto \beta \fto (\alpha, \beta)}$}
\end{figure}

\paragraph{Typechecking the Pair Function}
The pair function $\lam{x\;y} (x, y)$ takes two arguments, and returns a pair of the two values. Here, we type check it against its correct type $\alltype{\alpha\;\beta}{\alpha \fto \beta \fto (\alpha, \beta)}$. 

\begin{enumerate}
    \item Here, we begin typechecking with the \AllIntro rule to introduce $\forall \alpha$. We do this by adding $\alpha$ to the initially empty context. We then check the function against the type without the $\forall \alpha$: $\alltype{\beta}{\alpha \fto \beta \fto (\alpha, \beta)}$. The output of this checking is then split into 3 parts: everything before the $\alpha$, the $\alpha$ itself, and the bits after the $\alpha$. Our output context is everything in $\Delta$ before the alpha, which is nothing. 
    \item We apply the same rule as above, but we are introducing $\forall \beta$. We then check the function against ${\alpha \fto \beta \fto (\alpha, \beta)}$
    Our output context for this rule is everything in $\Gamma$ before the $\beta$: only $\alpha$.
    \item We then start to unwrap the abstractions. We strip the abstraction over $x$ from the expression, leaving us with ($\lam{y} (x, y)$). We then add $(x:\alpha)$ to our context, and then progress by checking the remaining part of the expression against ${\beta \fto (\alpha, \beta)}$. Our output context is $\Gamma$. 
    \item Same as above, with $y$ against $\beta$. We unwrap the abstraction over $y$ to give us $(x, y)$. We then check this against ${(\alpha, \beta)}$. The output context is $\Gamma$.
    \item We check $(x, y)$ against the type $(\alpha, \beta)$. To check this, we check $x$ against $\alpha$ and $y$ against $\beta$. The output context is $\Gamma$.
    \item To check $x$ against type $\alpha$ we synthesise the type of $x$ ([9]: trivial, as its in the context). We then check this against $\alpha$, and a check of $\alpha$ against $\alpha$ ([10]) trivially passes.
    \item Same as above with $y$ against $\beta$. The output context is $\Gamma$
\end{enumerate}



% \section{Typechecking an Expression Involving Lists}
% We shall attempt to use the algorithm to check the type of $\text{Cons}\;1\;x$ against $Int \fto List\;Int$. This derivation will skip some trivial subtyping/instantiation, but it should serve as a demonstration of how more complex checking works. 

% \begin{mathpar}
% T_{Cons} = \alltype{\alpha}{\alpha \fto List\; \alpha \fto List\;\alpha}\\
% \Gamma_0 = T_{Cons}\and
% \Gamma_1 = \Gamma_0, \ahat, \bhat, x : \ahat \\

% \Infer{\Sub[1]}
%     {
%         [2]
%         \\
%         \subjudg{\Theta}{[\Theta]A}{[\Theta]B}{\Delta}
%     }
%     {\chkjudg{\Gamma}{\lam{x} Cons\;1\;x}{List\;Int}{\Delta}}

% \\

% {\Infer{{\!\ArrIntroSyn}[2]}
%     {
%     \chkjudg{\Gamma_1}{Cons\;1\;x}{\bhat}{\Delta, x : \ahat, \Theta}
%     }
%     {{\synjudg{\Gamma_0}{\lam{x} Cons\;1\;x}{\ahat \arr \bhat}{\Delta}}}

%     \Infer{\!\ArrElim[3]}
%         {
%         {
%             [4]{\synjudg{\Gamma_1}{Cons\,1}{C}{\Delta}}
%         }
%         \\
%         \appjudg{\Gamma_1}{x}{[\Gamma_1]A}{C}{\Delta}
%         }
%         {\synjudg{\Gamma}{Cons\,1\,x}{C}{\Delta}}
% }

% \\
% \Infer{\!\ArrElim[4]}
%     {
%     {
%     \Infer{\Var}
%         {(Cons : T_{Cons}) \in \Gamma_1}
%         {\synjudg{\Gamma_1}{Cons}{T_{Cons}}{\Gamma_1}}
%     }
%     \\
%     {
%      \Infer{\AllApp}
%         {\appjudg{\Gamma_1,\ahat}{1}{(\ahat \fto List\; \ahat \fto List\;\ahat)}{C}{\Delta}}
%         {\appjudg{\Gamma_1}{1}{\alltype{\alpha}{\alpha \fto List\; \alpha \fto List\;\alpha}}{C}{\Delta}}
%     }
%     }
%     {\synjudg{\Gamma_1}{Cons\,1}{C}{\Delta}}
% \end{mathpar}


