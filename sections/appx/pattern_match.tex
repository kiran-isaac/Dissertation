\chapter{Pattern Matching Algorithm}
\label{appx:pattern_match}
A pattern consists of only:
\begin{itemize}
    \item An application
    \item A literal
    \item A pair
    \item An identifier: could be a wildcard, a constructor.
\end{itemize}

Below is the algorithm for each of these cases, as well as the top level pattern matching algorithm. 

\begin{figure}[h]
    \begin{lstlisting}[language=Rust_boxed]
fn get_redex_from_match(match_expression) -> Option<RedexContractionPair> {
    // get the expression being matched
    let expr = match_expression.to_be_matched
    // Iterate through all the patterns and their resulting expressions
    for ((pattern, resulting_expr) in match_expression.cases) {
        let result = pattern_match(expr, pattern);
        if (result == Refute) {
            // If refuted, then we can safely consider next pattern
            continue
        }
        if (result == Unknown) {
            // We get the reduction option for the expression
            // as we cannot refute this pattern
            return get_redex(expr)
        }
        if (result == Success(bindings)) {
            return Some(RedexContractionPair {
                from: match_expression,
                to: resulting_expr.substitute(bindings),
                reduction_message: "Match to pattern" + pattern.to_string()
            })
        }
    }
    // Refuted all patterns
    return None
}
\end{lstlisting}
    \caption{The algorithm for getting the redex-contraction pair from a match expression. If we sucessfully match, the result will be the expression corresponding to the matching pattern. If we cannot match expressions}
    \label{fig:all_pattern_list_iterate}
\end{figure}

\begin{figure}[h]
    \begin{lstlisting}[language=Rust_boxed]
fn pattern_match(expr, pattern) -> MatchResult {
    if (pattern is identifier) {match_against_identifier(expr, pattern)}
    if (pattern is a pair) {match_against_pair(expr, pattern)}
    if (pattern is an app) {match_against_application(expr, pattern)}
    if (pattern is a literal and expr is a literal) {
        if (expr.to_string() == pattern.to_string()) 
            return Success([])
        } else {
            return Refute
        }
    }
    return Unknown
}
\end{lstlisting}
    \caption{The algorithm for matching an expression against a pattern}
    \label{fig:pattern_list_top_level}
\end{figure}

\begin{figure}[h]
    \begin{lstlisting}[language=Rust_boxed]
fn match_against_identifier(expr, pattern) -> MatchResult {
    if (pattern is "_") {
        // Succeed but dont bind anything
        Success([])
    }
    if (pattern is a lowercase identifier) {
        // We suceed as a lowercase ID is a wildcard, and we must add to 
        // our list of bindings the fact that the named wildcard now has a 
        // value: the expr
        Success([(pattern.string, expr)])
    }
    if (pattern is a constructor (i.e. is uppercase)) {
        if (expr is also a constructor with the same name) {
            return Success([])
        }
        if (expr is an application) {
            // `Head' refers to the recursive front of an application. For 
            // instance, The head of (Left ((Cons x) xs)) would be Left.
            if (the head of expr is a constructor) {
                // We can refute, as constructors never evaluate, so the 
                // structure of the expression will never be the same as
                // the pattern. 
                return Refute
            } else {
                // Otherwise further evaluation might lead to a pattern 
                // that matches this constructor so we cant refute yet
                return Unknown
            }
        }
        return Unknown;
    }
}
    \end{lstlisting}
    \caption{The algorithm for matching an expression against a pattern that is an identifier in rust like pseudocode}
    \label{fig:pattern_list_id}
\end{figure}

\begin{figure}[h]
    \begin{lstlisting}[language=Rust_boxed]
fn match_against_pair(expr, pattern) -> MatchResult {
    if (expr is also a pair) {
        let first = pattern_match(expr.first, pattern.first);
        let second = pattern_match(expr.second, pattern.second);

        // Propogate refute and unknown
        if (first == Unknown || second == Unknown) {
            return Unknown;
        }
        if (first == Refute || second == Refute) {
            return Refute;
        }
        // first and second have suceeded, return both sets of bindings
        return Success(first.bindings + second.bindings)
    }
    if (expr is an application) {
        if (the head of expr is a constructor) {
            return Refute
        } else {
            return Unknown
        }
    }
    if (expr is a literal || expr is an abstraction 
        || expr is an uppercase identifier) {return Refute}
    
    return Unknown // catchall: only `match'
}
\end{lstlisting}
    \caption{The algorithm for matching an expression against a pattern that is a pair in rust like pseudocode. See \ref{fig:pattern_list_id} for more detail about the `expr is application' case}
    \label{fig:pattern_list_pair}
\end{figure}

\begin{figure}[h]
    \begin{lstlisting}[language=Rust_boxed]
fn match_against_application(expr, pattern) -> MatchResult {
    if (expr is also an application) {
        let func = pattern_match(expr.func, pattern.func);
        let arg = pattern_match(expr.arg, pattern.arg);

        // Propogate refute and unknown
        if (func == Unknown || arg == Unknown) {
            return Unknown;
        }
        if (func == Refute || arg == Refute) {
            return Refute;
        }
        // func and arg have suceeded, return both sets of bindings
        return Success(func.bindings + arg.bindings)
    }
    if (expr is a literal || expr is a pair || expr is an abstraction 
        || expr is an uppercase identifier) {return Refute}
    return Unknown
}
\end{lstlisting}
    \caption{The algorithm for matching an expression against a pattern that is a pair in rust like pseudocode. See \ref{fig:pattern_list_id} for more detail about the `expr is application' case}
    \label{fig:pattern_list_app}
\end{figure}
