\chapter{Cycle 2 - Types and Pattern Matching}
This second cycle was about extending the language with more complicated features. 

\section{Requirements Analysis}
I presented the proof of concept project 

\section{Design}
\subsection{Type System}
We must have types representing integers and booleans in our language, if we are to effectively check the validity of expression containing their respective literals. 

Many languages, including Haskell, also have Algebraic Data Types allowing us to "Compose" other data types. Algebraic data types are isomorphic to an algebraic expression consisting of sums and products of their constituent types. An example of a product type is the tuple \((Int,Bool)\) which is isomorphic to \(Int \times Bool\). Most languages have product types, which often take the form of structs or tuples. 
An example of a sum type is the Haskell syntax tagged union:

\noindent\verb!"data Shape = Circle Int | Rectangle Int Int"!, which is isomorphic to the type \(Int + (Int \times Int)\). 

We will now consider generic data structures, such as a lists that can hold any value of type $a$, written as \(List\;a\)". Here, \(List\) is not a type in itself, but it represents a constructor that takes a type, and returns a concrete type. We could write this as "$Type \rightarrow Type$", indicating that it behaves like a function, but at the type level rather than the value level. If we were to apply the constructor \(List\) to the concrete type \(Int\), the resulting type would be \(List \;Int\). 

Polymorphism as described here is first order polymorphism, as opposed to higher-order polymorphism (also known as higher-kinded polymorphism) where a type can abstract over a type that abstracts over a type \cite{pierce2002types}. An example of a function that is higher-order polymorphic is a function that takes a function, and then applies it to two differently typed values:
\begin{lstlisting}
applyToBoth f x y = (f x, f y)
\end{lstlisting}
If \verb|f| is to be applied to any type, it must have a type \(\forall a. a\rightarrow a\). This means the type of the function \verb|applyToBoth| must be \(\forall a \;b.(\forall c \rightarrow c) \rightarrow a \rightarrow b \rightarrow (a, b)\)

This requires the ability to parse expressions with nested \verb|foralls|, as well as support during type inference for higher-kinded types. We do not think that this is a priority for the system, as 
These first-order polymorphic type constructors would be useful to have in \ac{SFL}, with one example of their utility being defining the polymorphic function "\verb|length :: List a -> Int|" which should work regardless of what type the list is over. Higher order polymorphism is less important
The "Type of a Type" is known as its \emph{kind} \cite{pierce2002types}. Another example is the type representing "Either left or right: \(Either \;a\;b\)", that can be defined with its constructors as \verb!data Either a b = Left a | Right b!. "Either" is a type constructor with the kind "Type -> Type -> Type", meaning it takes two concrete types and returns a concrete type. 
Higher-kinded types are types where there are parenthesis in a kind expression, not including the implicit ones implying right associativity: 
"\verb|Type -> Type -> Type|" is implicitly "\verb|Type -> (Type -> Type)|"
We can avoid thinking about kinds by enforcing that a type constructor is always given the correct number of arguments

Supporting tagged unions and tuples in the \ac{SFL} type system would massively increase the ease of writing complex programs. It would also allow for complex data structures such as trees and lists. 
Type names, as well as constructor names, start with uppercase letters in Haskell. This allows them to be easily differentiated from type variables, as well as regular variables. 
In the below definition of the \ac{SFL} type system, we define \(\Alpha\) as the set of valid type names starting with uppercase letters defined by \(Id\), and \(\alpha\) as the set of valid type variable names defined by \(id\). 

\begin{syntax}[Types in \ac{SFL}]
(Inbuilts): \(B::=Int\mid Bool\)\newline
(Monotype): \(\tau, \sigma ::= \alpha \mid B \mid \tau \rightarrow \sigma \mid (\tau, \sigma) \mid \Alpha \;T, U,...\)\newline
(Alias): \(A ::= Id = T\)\newline
(Type): \(T, U ::= \alpha \mid B \mid T \rightarrow U \mid (T, U) \mid \forall a. T \mid \Alpha \;T, U,...\)
\end{syntax}
Note that our type constructor application definition above is more permissive than is correct, as it does not enforce correct arity. This can be handled by the parser maintaining the context of the arity of each type. It can then be double checked for debugging purposes via an assertion in the type checker. 

\subsection{Match}
To support different execution based on a condition, we must have some structure that can differentiate between values \ref{design:values}. 

\subsection{Syntax Sugar}
