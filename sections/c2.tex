\chapter{Cycle 2 - Types and Pattern Matching}
The aim of this cycle was to develop the more technical aspects of the project. In this cycle, I moved away from the autoethnographic (\ref{sec:c1_autoethnography}) approach, where most of my requirements came from within, to an externally motivated approach. The requirements for this cycle were motivated by my client meeting (\ref{eval:c1}). 

\section{Requirements Analysis}
Going into this cycle, we have a proof of concept system and an idea for how the system will look. We also have a 
The client had many valuable insights into what features would be valuable to add. Below is a list of features identified to add to the system in this cycle. 

\paragraph{Language}
\begin{itemize}
    \item Pattern Matching
    \item User definable data types: tagged unions
    \item Pairs
\end{itemize}

\section{Design}
\subsection{Language Changes}
\subsubsection{Type System}
We must have types representing integers and booleans in our language, if we are to effectively represent the type of expression containing their respective literals. We also want our type system to be able to express functions. We also want polymorphism in our type system, as rewriting functions many times for different data types makes programs more verbose. 

Allowing for algebraic user defined data types in a similar manner to Haskell would make the language much more expressive and much more powerful, as well as bringing it closer to Haskell. Supporting tagged unions and tuples in the \ac{SFL} type system would massively increase the ease of writing complex programs. It would also allow for complex data structures such as trees and lists. 

Type names, as well as constructor names, start with uppercase letters in Haskell. This allows them to be easily differentiated from type variables, as well as regular variables. 

First-order polymorphic type constructors would be useful to have in \ac{SFL}, with one example of their utility being defining the polymorphic function "\verb|length :: List a -> Int|" which should work regardless of what type the list is over.

\begin{figure}[h]
  \centering
  \[
      \begin{array}{llcl}
      \text{Types} & A, B, C & \bnfas &
            \Inttype \bnfalt \Booltype \bnfalt \alpha \bnfalt 
            \alltype{\alpha}{A} \bnfalt A \arr B \bnfalt
            % \\[1pt] &&&\!\!\!\;\; \TypeAlias{A}{B} \bnfalt  
            (A, B) \bnfalt
            \Unionname[A_1, \dots, A_n]
      \\[2pt]
      \text{Monotypes} & \tau,\sigma & \bnfas &
            \Inttype \bnfalt \Booltype \bnfalt \alpha \bnfalt \ahat \bnfalt 
            A \arr B \bnfalt (A, B)
            % (A, B) \bnfalt \TypeAlias{A}{B}
        \\[2pt]
      \end{array}
  \]
  
  \captionsetup{justification=centering}\caption{Syntax of types and monotypes. Note that this is the external definition: as seen by the users of SFL. See \ref{fig:tc_types} for the extra type system structures required internally for the typechecker}
  \FLabel{fig:alg-syntax}
\end{figure}

We can avoid thinking about kinds by enforcing that a type constructor is always given the correct number of arguments

% \begin{syntax}[Types in \ac{SFL}]
% (Inbuilts): \(B::=Int\mid Bool\)\newline
% (Monotype): \(\tau, \sigma ::= \alpha \mid B \mid \tau \rightarrow \sigma \mid (\tau, \sigma) \mid \Alpha \;T, U,...\)\newline
% (Alias): \(A ::= Id = T\)\newline
% (Type): \(T, U ::= \alpha \mid B \mid T \rightarrow U \mid (T, U) \mid \forall a. T \mid \Alpha \;T, U,...\)
% \end{syntax}
Note that our type constructor application definition above is more permissive than is correct, as it does not enforce correct arity. This can be handled by the parser maintaining the context of the arity of each type. It can then be double checked for debugging purposes via an assertion in the type checker. 

\subsubsection{Match}
To support different execution based on a condition, we must have some structure that can differentiate between values \ref{design:values}. 

\subsubsection{Syntax Sugar}

\subsection{UI}
I completely redesigned the UI based on the clients feedback, as well as some other factors. See \ref{fig:screenshot_cycle_1_end} for the current state. See \ref{fig:screenshot_figma1} and \ref{fig:screenshot_figma2} for screenshots of the new design. 

This design was meant to be a work in progress but it looks quite similar to the final release of the product (Screenshots \ref{fig:screenshot_final_dark}, \ref{fig:screenshot_final_light}, \ref{fig:screenshot_final_dark_prelude_free} and \ref{fig:screenshot_final_light_prelude_free}). Before implementing this design, I discussed this design with the Advanced Focus Group (see \ref{ref:afg_figma}) and they were much more positive about this UI than the existing one 

\section{Implementation}
\subsection{The Parser}
We must make some changes to the parser to include these new features.

\subsubsection{Parsing Match Statements}
An example of using a match statement follows:
\begin{verbatim}
lengthIsAtLeast2 list = match list {
  | Cons x (Cons y xs) -> true
  | _ => false
}
\end{verbatim}

The algorithm used for parsing match statements is:
\begin{itemize}
    \item Consume the "match" keyword.
    \item Parse the expression matched over
    \item Consume an open brace
    \item While the next token isn't a close brace: \begin{itemize}
        \item Parse a pattern (\ref{impl:parsing_patterns}).
        \item Consume a right arrow
        \item Parse an expression
    \end{itemize}
    \item Consume a close brace
\end{itemize}
Following this, a match node is created, where the \verb|children| vector is set appropriately with the pattern and expressions.

\paragraph{Patterns}
\label{impl:parsing_patterns}
A pattern must be a value \ref{design:values}; a pattern must not contain anything that can be reduced. It would be nonsensical to have a situation where we had a pattern not in normal form such as \verb|1 + 1| and the expression to be matched was \verb|2|. 

To parse a pattern, we may use the same techniques as parsing an expression, with a few differences:
\begin{itemize}
    \item Disallowing abstractions
    \item Identifiers must be either:
    \begin{itemize}
        \item Unbound lowercase variables
        \item Underscore (\verb|_|) representing a wildcard pattern
        \item A bound uppercase variable (a constructor)
    \end{itemize}
\end{itemize}

\subsection{Types}

Rust allows us to represent our types (see \ref{fig:tc_types} for the definition of the type system), quite easily using Enums. Rust's Enums are an example of algebraic data types, and are therefore very useful for defining our own algebraic data type system. See \ref{fig:type_lst} for the listing. 

\begin{figure}[ht]
    \centering
    \begin{tabular}{c}
    \begin{lstlisting}[language=Rust]
pub enum Primitive {
    Int64,
    Bool,
}

pub enum Type {
    Unit,
    Primitive(Primitive),
    Function(Box<Type>, Box<Type>),
    TypeVariable(String),
    Forall(String, Box<Type>),
    Product(Box<Type>, Box<Type>),
    Union(String, Vec<Type>),
    Alias(String, Box<Type>),

    Existential(usize),
}
    \end{lstlisting}
    \end{tabular}
    \caption{The Rust code listing for the definition of types. Existential is separated as it is more of an implementation detail than a part of the type system}
    \label{fig:type_lst}
\end{figure}

We must use \verb|Box<Type>|, which represents a pointer to a heap allocated object, otherwise it would be impossible to calculate the size of \verb|Type|, as it could be infinitely large with it containing another \verb|Type| recursively. \verb|Box<Type>| however has known size: the size of a pointer in the target architecture. We also define Existential, as an implementation detail needed for the type checker. 

\subsubsection{Methods on Types}
Below are a selection of the more important or interesting methods implemented on Types.

\paragraph{Substitution of type variables} We may wish to set a type variable to another type. For instance, if given the type expression \(T \; U\) where \(T\) and \(U\) are types, and we know that one of the constructors of \(T\) is of generic type \(\forall a.a \rightarrow T \; a\), the type of the constructor for this type should be \(U \rightarrow T \; U\). We have `instantiated' the type variable \(a\) to be \(U\) by substituting \(a\) with \(U\) throughout the expression, and removing the \(\forall a\). This is required for the type checker. 

\paragraph{To String} We will frequently wish to display types as strings for debugging purposes. 

\subsection{The Type Checker}
The type checker will be bidirectional, and will follow and algorithm largely based on the one in \cite{completebidir}. The quote that follows is from this paper, and describes bidirectional type checking and its merits:
\begin{quote}
`Bidirectional typechecking, in which terms either synthesize a type or are checked against a known type, has become popular for its scalability ..., its error reporting, and its relative ease of implementation.'
\end{quote}
\noindent It was the `relative ease of implementation' that attracted me to bidirectional type checking. After running the algorithm by hand to convince myself the algorithm works on checking id ($\lambda x. x$) against type $\forall a.a\fto a$ (photo of whiteboard derivation included in submission [TODO: is this the right way to cite this]), I modified their algorithm to add my extra types (The inbuilt types $Int$, $Bool$, as well as the \Uniontype\ and \Producttype\ types) and my extra expression syntax structures (literals, match, pairs. Not including assignment and modules as these are not part of expression syntax). \ref{fig:tc_types} shows the type system, including the typechecker implementation details, as well as the unmodified context structure, which keeps track of the state of the typechecker as it progresses recursively through the type system. \ref{fig:ctx_subst} shows the modified algorithm for substituting all of the information in a context into a type. 

\begin{figure}[h]
  \centering
  \begin{minipage}{\textwidth}
  \[
      \begin{array}{llcl}
      % \\[4pt]
      \text{Types} & A, B, C & \bnfas &
            \Inttype \bnfalt \Booltype \bnfalt \alpha \bnfalt \ahat \bnfalt 
            % \\[1pt] &&&\!\!\!\;\;\;
            \alltype{\alpha}{A} \bnfalt A \arr B \bnfalt
            % \\[1pt] &&&\!\!\!\;\; \TypeAlias{A}{B} \bnfalt  
            % \\[1pt] &&&\!\!\!\;\;
            (A, B) \bnfalt
            % \\[1pt] &&&\!\!\!\;\;
            \Unionname[A_1, \dots, A_n]
      \\[2pt]
      \text{Monotypes} & \tau,\sigma & \bnfas &
            \Inttype \bnfalt \Booltype \bnfalt \alpha \bnfalt \ahat \bnfalt 
            % \\[1pt] &&&\!\!\!\;\;
            \tau \arr \sigma \bnfalt (\tau, \sigma) \bnfalt \Unionname[\tau_1, \dots, \tau_n]
            % (A, B) \bnfalt \TypeAlias{A}{B}
        \\[2pt]
      \text{Contexts} & \Gamma, \Delta, \Theta & \bnfas &
                  \cdot
                  \bnfalt \Gamma, \alpha 
                  \bnfalt \Gamma, x:A
                  % \\[1pt] &&&\!\!\!\;\;
                  \bnfalt \Gamma, \ahat
                  \bnfalt \Gamma, \hypeq{\ahat}{\tau}
                  % \\[1pt] &&&\!\!\!\;\;
                  \bnfalt \Gamma, \MonnierComma{\ahat}
      % \\[2pt] % No need as I am not proving completeness
      % \text{Complete Contexts}     & \Omega & \bnfas &
      %             \cdot
      %             \bnfalt    \Omega, \alpha
      %             \bnfalt    \Omega, x:A
      %             \\[1pt] &&&\!\!\!
      %             \bnfalt    \Omega, \hypeq{\ahat}{\tau} 
      %             \bnfalt    \Omega, \MonnierComma{\ahat}
      \end{array}
  \]
  
  \captionsetup{justification=centering}\caption{Syntax of types, monotypes, and contexts as seen by the typechecker. The definition of types differ slightly from the definition offered in figure \ref{fig:sfl_types_no_exst}, as we include existential type variables ($\ahat$) that can not actually be created by users. They are an implementation detail required for the type checking algorithm}
  \FLabel{fig:tc_types}

  
  \end{minipage}
  \hfill
  \begin{minipage}{\textwidth}
    \centering
      \[
  \begin{array}[t]{l@{~}c@{~}ll}
      %
      [\Gamma]\Inttype  & = & \Inttype &\\{}

      [\Gamma]\Booltype & = & \Booltype &\\{}

      [\Gamma]\alpha    & = & \alpha &\\{}
      % [\Gamma]\unitty   & = &   \unitty &
      % \\[1pt]
      \big[\Gamma[\hypeq{\ahat}{\tau}]\big] \ahat
               & = &   \big[\Gamma[\hypeq{\ahat}{\tau}]\big]\tau &
      \\[2pt]
      \big[\Gamma[\ahat]\big]\ahat   & = &   \ahat &
      \\[2pt]
      [\Gamma](A \arr B)   & = &
          ([\Gamma]A) \arr ([\Gamma]B) &
      \\{}
      [\Gamma](\alltype{\alpha} A)
         & = & 
         \alltype{\alpha} [\Gamma]A &
      \\[2pt]
      [\Gamma](A, B) 
         & = &
         ([\Gamma]A, [\Gamma]B)
      \\[2pt]
      [\Gamma]\Unionname[A_1,\dots, A_n]
         & = &
         \Unionname[[\Gamma]A_1,\dots, [\Gamma]A_n]
  \end{array}
  \]
  \captionsetup{justification=centering}\caption{Applying a context, as a substitution, to a type}
  \FLabel{fig:ctx_subst}
  \end{minipage}
\end{figure}


The full typechecking algorithm is listed here below. Most of these rules are untouched, the ones that I added or modified are \colorbox{myTcRuleColour}{highlighted}.

\begin{figure*}[htbp]
  \judgbox{\subjudg{\Gamma}{A}{B}{\Delta}}%
     {Under input context $\Gamma$,
       type $A$ is a subtype of $B$, with output context $\Delta$}
  \begin{mathpar}
  \Infer{\SubVar}
              { }
              {\subjudg{\Gamma[\alpha]}{\alpha}{\alpha}{\Gamma[\alpha]}}
  \and
  \Infer{\MyTCRule{\Intsubrulename}}
            { }
            {\subjudg{\Gamma}{\Inttype}{\Inttype}{\Gamma}}
              \and
  \Infer{\MyTCRule{\Boolsubrulename}}
            { }
            {\subjudg{\Gamma}{\Booltype}{\Booltype}{\Gamma}}
  \\
  % \Infer{\MyTCRule{\LAliassubrulename}}
  %           { \subjudg{\Gamma}{C}{B}{\Delta}  }
  %           { \subjudg{\Gamma}{\TypeAlias{A}{C}}{B}{\Delta} }
  % \and
  %   \Infer{\MyTCRule{\RAliassubrulename}}
  %           { \subjudg{\Gamma}{A}{C}{\Delta}  }
  %           { \subjudg{\Gamma}{A}{\TypeAlias{B}{C}}{\Delta} }
  % \\
  \Infer{\SubExvar}
            { }
            {\subjudg{\Gamma[\ahat]}{\ahat}{\ahat}{\Gamma[\ahat]}}
  \and
  \Infer{\SubArr}
            {\subjudg{\Gamma}{B_1}{A_1}{\Theta} \\
              \subjudg{\Theta}{[\Theta]A_2}{[\Theta]B_2}{\Delta}}
            {\subjudg{\Gamma}{A_1 \arr A_2}{B_1 \arr B_2}{\Delta}}
  \\
  \Infer{\SubAllL}
            {\subjudg{\Gamma, \MonnierComma{\ahat}, \ahat}
                     {[\ahat/\alpha]A}
                     {B}
                     {\Delta, \MonnierComma{\ahat}, \Theta}}
            {\subjudg{\Gamma}{\alltype{\alpha}{A}}{B}{\Delta}}
  \and
  \Infer{\SubAllR}
            {\subjudg{\Gamma, \alpha}{A}{B}{\Delta, \alpha, \Theta}}
            {\subjudg{\Gamma}{A}{\alltype{\alpha}{B}}{\Delta}}
  \\
  \Infer{\SubInstL}
            {
              \ahat \notin \FV{%
                A}
              \\
              \instjudg{\Gamma[\ahat]}{\ahat}{%
                   A}{\Delta}
            }
            {\subjudg{\Gamma[\ahat]}{\ahat}{A}{\Delta}}
  \and
  \Infer{\SubInstR}
            {
              \ahat \notin \FV{%
                A}
              \\
              \instjudgr{\Gamma[\ahat]}{\ahat}{%
                A}{\Delta}
            }
            {\subjudg{\Gamma[\ahat]}{A}{\ahat}{\Delta}}
  \\
  \Infer{\MyTCRule{\Pairsubrulename}}
    {\subjudg{\Gamma}{A_1}{A_2}{\Theta} \\
              \subjudg{\Theta}{B_1}{B_2}{\Delta}}
    { \subjudg{\Gamma}{(A_1, B_1)}{(A_2, B_2)}{\Delta} }
      \\
  \Infer{\MyTCRule{\Unionsubrulename}}
    {(\subjudg{\Gamma_i}{A_i}{B_i}{\Gamma_{i+1}})\;\tcforin{i}{[1, 2, \dots, n]}}
    {\subjudg{\Gamma_1}{\Unionname[A_1, A_2, \dots, A_n]}{\Unionname[B_1, B_2, \dots, B_n]}{\Gamma_{n+1}}}
  \end{mathpar}  
  \caption{Algorithmic subtyping. The rules with \colorbox{myTcRuleColour}{highlighted} names are my additions, the rest are unchanged from \cite{completebidir}}
  \FLabel{fig:alg-subtyping}
\end{figure*}


\begin{figure*}[h]
      \judgbox{\instjudg{\Gamma}{\ahat}{A}{\Delta}}%
         {Under input context $\Gamma$,
           instantiate $\ahat$ such that $\ahat \subtype A$, 
           with output context $\Delta$}
      \begin{mathpar}
        \Infer{\InstLSolve}
                { \Gamma \entails \tau} %
                { \instjudg{\Gamma, \ahat, \Gamma'}
                            {\ahat}
                            {\tau}
                            {\Gamma, \hypeq{\ahat}{\tau}, \Gamma'}
                 }
        \and
        \Infer{\InstLReach}
                { }
                {\instjudg{\Gamma[\ahat][\bhat]}
                            {\ahat}
                            {\bhat}
                            {\Gamma[\ahat][\hypeq{\bhat}{\ahat}]}}
        \and
        \Infer{\InstLArr}
                {\instjudgr{\Gamma[\ahat_2, \ahat_1, \hypeq{\ahat}{\ahat_1 \arr \ahat_2}]}
                            {\ahat_1}
                            {A_1}
                            {\Theta} \\
                 \instjudg{\Theta}
                            {\ahat_2}
                            {[\Theta]A_2}
                            {\Delta}}
                {\instjudg{\Gamma[\ahat]}
                            {\ahat}
                            {A_1 \arr A_2}
                            {\Delta}}
        \and
        \Infer{\InstLAllR}
              {\instjudg{\Gamma[\ahat], \beta}{\ahat}{B}{\Delta, \beta, \Delta'}}
              {\instjudg{\Gamma[\ahat]}{\ahat}{\alltype{\beta}{B}}{\Delta}}
      \end{mathpar}    
    %
    \\[-1.5ex]
    %
      \judgbox{\instjudgr{\Gamma}{\ahat}{A}{\Delta}}
         {Under input context $\Gamma$,
           instantiate $\ahat$ such that $A \subtype \ahat$,
           with output context $\Delta$}
      \small\begin{mathpar}
        \Infer{\InstRSolve}
                { \Gamma \entails \tau}
                { \instjudgr{\Gamma, \ahat, \Gamma'}
                            {\ahat}
                            {\tau}
                            {\Gamma, \hypeq{\ahat}{\tau}, \Gamma'}
                 }
        \and
        \Infer{\InstRReach}
                { }
                {\instjudgr{\Gamma[\ahat][\bhat]}
                           {\ahat}
                           {\bhat}
                           {\Gamma[\ahat][\hypeq{\bhat}{\ahat}]}}
        \and
        \Infer{\InstRArr}
              {\instjudg{\Gamma[\ahat_2, \ahat_1, \hypeq{\ahat}{\ahat_1 \arr \ahat_2}]}
                        {\ahat_1}
                        {A_1}
                        {\Theta} \\
                 \instjudgr{\Theta}
                           {\ahat_2}
                           {[\Theta]A_2}
                           {\Delta}}
              {\instjudgr{\Gamma[\ahat]}
                         {\ahat}
                         {A_1 \arr A_2}
                         {\Delta}}
        \and 
        \Infer{\InstRAllL}
              {\instjudgr{\Gamma[\ahat], \MonnierComma{\bhat}, \bhat}{\ahat}{[\bhat/\beta]B}{\Delta, \MonnierComma{\bhat}, \Delta'}}
              {\instjudgr{\Gamma[\ahat]}{\ahat}{\alltype{\beta}{B}}{\Delta}}
      \end{mathpar}

\caption{Instantiation}
\FLabel{fig:instantiation}
\end{figure*}
\begin{figure*}[ht]
  \judgbox{\chkjudg{\Gamma}{e}{A}{\Delta}}%
     {Under input context $\Gamma$, $e$ checks against input type $A$, 
     with output context $\Delta$} \\[1ex]
  \judgbox{\synjudg{\Gamma}{e}{A}{\Delta}}%
     {Under input context $\Gamma$, $e$ synthesizes output type $A$,
       with output context $\Delta$} \\[0.5ex]
  \judgbox{\appjudg{\Gamma}{e}{A}{C}{\Delta}}%
     {Under input context $\Gamma$, applying a function of type $A$ to $e$ synthesizes type $C$, \\ with output context $\Delta$} \\
  \begin{mathpar}
    \Infer{\MyTCRule{\Intcheckrulename}}
        { }
        {\chkjudg{\Gamma}{Int Literal}{\Inttype}{\Gamma}}
    \and
    \Infer{\MyTCRule{\Intsynthrulename}}
        { }
        {\synjudg{\Gamma}{Int Literal}{\Inttype}{\Gamma}}
    \\
    \Infer{\MyTCRule{\Boolcheckrulename}}
        { }
        {\chkjudg{\Gamma}{Bool Literal}{\Booltype}{\Gamma}}
    \and
    \Infer{\MyTCRule{\Boolsynthrulename}}
        { }
        {\synjudg{\Gamma}{Bool Literal}{\Booltype}{\Gamma}}
    \\
    \Infer{\MyTCRule{\Paircheckrulename}}
        {\chkjudg{\Gamma}{e_1}{A}{\Theta} \\ \chkjudg{\Theta}{e_2}{B}{\Delta}}
        {\chkjudg{\Gamma}{(e_1, e_2)}{(A, B)}{\Delta}}
        \and 
      \Infer{\MyTCRule{\Pairsynthrulename}}
        {\synjudg{\Gamma}{e_1}{A}{\Theta} \\ \synjudg{\Theta}{e_2}{B}{\Delta}}
        {\synjudg{\Gamma}{(e_1, e_2)}{(A, B)}{\Delta}}
     \\
     \Infer{\Var}
          {(x : A) \in \Gamma}
          {\synjudg{\Gamma}{x}{A}{\Gamma}}
      \and
    %  \Infer{\MyTCRule{\Aliasrulename}}
    %     {\chkjudg{\Gamma}{e}{B}{\Delta}}
    %     {\chkjudg{\Gamma}{e}{[A : B]}{\Delta}}
    % \and
     \Infer{\Sub}
          {\synjudg{\Gamma}{e}{A}{\Theta}
            \\
            \subjudg{\Theta}{[\Theta]A}{[\Theta]B}{\Delta}
          }
          {\chkjudg{\Gamma}{e}{B}{\Delta}}
      \\ 
     \def\CompactJudgments{1}   %
     \Infer{\AllIntro}
           {\chkjudg{\Gamma, \alpha}{e}{A}{\Delta, \alpha, \Theta}
           }
           {\chkjudg{\Gamma}{e}{\alltype{\alpha}{A}}{\Delta}}
     \and
     \Infer{\AllApp}
            {\appjudg{\Gamma,\ahat}{e}{[\ahat/\alpha]A}{C}{\Delta}}
            {\appjudg{\Gamma}{e}{\alltype{\alpha}{A}}{C}{\Delta}}
    \and
     \Infer{\!\ArrIntro}
          {\chkjudg{\Gamma, x : A}{e}{B}{\Delta, x : A, \Theta}
          }
          {\chkjudg{\Gamma}{\lam{x} e}{A \arr B}{\Delta}}
     \\
\def\CompactJudgments{0}   %
     \Infer{%
        {\!\ArrIntroSyn}
         }
           { \chkjudg{\Gamma, \ahat, \bhat, x : \ahat}{e}{\bhat}{\Delta, x : \ahat, \Theta}
           }
           {{\synjudg{\Gamma}{\lam{x} e}{\ahat \arr \bhat}{\Delta}}}
     \and
     \Infer{\!\ArrElim}
           {\synjudg{\Gamma}{e_1}{A}{\Theta}
             \\
             \appjudg{\Theta}{e_2}{[\Theta]A}{C}{\Delta}
           }
           {\synjudg{\Gamma}{e_1\,e_2}{C}{\Delta}}
      \\
\def\CompactJudgments{0}
      %
      \Infer{\SolveApp}
            {\chkjudg{\Gamma[\ahat_2, \ahat_1, \hypeq{\ahat}{\ahat_1 \arr \ahat_2}]}{e}{\ahat_1}{\Delta}}
            {\appjudg{\Gamma[\ahat]}{e}{\ahat}{\ahat_2}{\Delta}}
      \and
      \Infer{\ArrApp}
            {\chkjudg{\Gamma}{e}{A}{\Delta}}
            {\appjudg{\Gamma}{e}{A \arr C}{C}{\Delta}}
      \\
      \Infer{\MyTCRule{\Matchsynthrulename}}
            {
            % Change to for all
            % Add footnote
            \synjudg{\Gamma}{e}{A}{\Theta_1} \\ 
            (\chkjudg{\Theta_i}{c_i}{A}{\Theta_{i+1}})\; \tcforin{i}{[1, 2, \dots, n]} \\
            \Delta_1 = \Theta_{n+1}, \ahat \\
            (\chkjudg{\Delta_{i}}{e_i}{\ahat}{\Delta_{i+1}})\; \tcforin{i}{[1, 2, \dots, n]}
            }
            {\synjudg{\Gamma}{\text{match e \{} c_1 \rightarrow e_1 \mid c_2 \rightarrow e_2 \mid \dots \mid c_n \rightarrow e_n \}}{\ahat}{\Delta_{n+1}}}
    
  \end{mathpar}
  \caption{Algorithmic typing. The rules with highlighted names are my additions, the rest are unchanged from \cite{completebidir}. }
  \FLabel{fig:alg-typing}
  \noindent\hrulefill
\end{figure*}


\section{Evaluation}
\label{ref:afg_figma}
The aims of this cycle were to develop the language as well as some of the other more technical features of this project. 

At the end of this cycle, I held a focus group with the following aims:

\begin{itemize}
    \item Discussing the language with programming language experts.
    \item
\end{itemize}