\chapter{User Testing}
\label{chap:evaluation}

This desired outcome of this project is an effective learning/teaching tool for functional languages. As such, user testing is vital for ensuring that the system is usable and intuitive, and therefore effective. I conducted user testing towards the end of the project to help with evaluation. The testing was conducted with sufficient time to make small changes based on the outcomes of user testing.

I tested my system in 3 separate ways. I held 3 focus groups with 12 people in total, all with varying levels of experience with functional programming. I also attended the universities `Testathon', where I got feedback from 19 people [TODO: Poster day].

\section{Testathon}
The testathon was a valuable opportunity to test my system at the midpoint of the project. At this point, I had implemented the following features:

\begin{itemize}
    \item Parsing and type checking for the following language features
    \begin{itemize}
        \item Literals
        \item Variables
        \item Abstraction 
        \item Application
        \item Some inbuilt binary operators 
        \item The inbuilt \verb|if|
        \item Polymorphism, with explicit \verb|foralls|
    \end{itemize}
\item A basic react frontend (see \ref{fig:screenshot_testathon}) with the following features
\begin{itemize}
    \item Beginning evaluation in `lazy' mode, or `free choice' mode. These were labelled unhelpfully as `Run 1' and `Run *' respectively, representing how it would either run in the mode where you get one option, or the mode where you get many `*'.
    \item Spawning a help menu. (see \ref{fig:screenshot_testathon_2})
\end{itemize}
\end{itemize}
