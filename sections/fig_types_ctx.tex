\begin{figure}[h]
  \centering
  \begin{minipage}{\textwidth}
  \[
      \begin{array}{llcl}
      % \\[4pt]
      \text{Types} & A, B, C & \bnfas &
            \Inttype \bnfalt \Booltype \bnfalt \alpha \bnfalt \ahat \bnfalt 
            % \\[1pt] &&&\!\!\!\;\;\;
            \alltype{\alpha}{A} \bnfalt A \arr B \bnfalt
            % \\[1pt] &&&\!\!\!\;\; \TypeAlias{A}{B} \bnfalt  
            % \\[1pt] &&&\!\!\!\;\;
            (A, B) \bnfalt
            % \\[1pt] &&&\!\!\!\;\;
            \Unionname[A_1, \dots, A_n]
      \\[2pt]
      \text{Monotypes} & \tau,\sigma & \bnfas &
            \Inttype \bnfalt \Booltype \bnfalt \alpha \bnfalt \ahat \bnfalt 
            % \\[1pt] &&&\!\!\!\;\;
            \tau \arr \sigma \bnfalt (\tau, \sigma) \bnfalt \Unionname[\tau_1, \dots, \tau_n]
            % (A, B) \bnfalt \TypeAlias{A}{B}
        \\[2pt]
      \text{Contexts} & \Gamma, \Delta, \Theta & \bnfas &
                  \cdot
                  \bnfalt \Gamma, \alpha 
                  \bnfalt \Gamma, x:A
                  % \\[1pt] &&&\!\!\!\;\;
                  \bnfalt \Gamma, \ahat
                  \bnfalt \Gamma, \hypeq{\ahat}{\tau}
                  % \\[1pt] &&&\!\!\!\;\;
                  \bnfalt \Gamma, \MonnierComma{\ahat}
      % \\[2pt] % No need as I am not proving completeness
      % \text{Complete Contexts}     & \Omega & \bnfas &
      %             \cdot
      %             \bnfalt    \Omega, \alpha
      %             \bnfalt    \Omega, x:A
      %             \\[1pt] &&&\!\!\!
      %             \bnfalt    \Omega, \hypeq{\ahat}{\tau} 
      %             \bnfalt    \Omega, \MonnierComma{\ahat}
      \end{array}
  \]
  
  \captionsetup{justification=centering}\caption{Syntax of types, monotypes, and contexts as seen by the typechecker. The definition of types differ slightly from the definition offered in figure \ref{fig:sfl_types_no_exst}, as we include existential type variables ($\ahat$) that can not actually be created by users. They are an implementation detail required for the type checking algorithm}
  \FLabel{fig:tc_types}

  
  \end{minipage}
  \hfill
  \begin{minipage}{\textwidth}
    \centering
      \[
  \begin{array}[t]{l@{~}c@{~}ll}
      %
      [\Gamma]\Inttype  & = & \Inttype &\\{}

      [\Gamma]\Booltype & = & \Booltype &\\{}

      [\Gamma]\alpha    & = & \alpha &\\{}
      % [\Gamma]\unitty   & = &   \unitty &
      % \\[1pt]
      \big[\Gamma[\hypeq{\ahat}{\tau}]\big] \ahat
               & = &   \big[\Gamma[\hypeq{\ahat}{\tau}]\big]\tau &
      \\[2pt]
      \big[\Gamma[\ahat]\big]\ahat   & = &   \ahat &
      \\[2pt]
      [\Gamma](A \arr B)   & = &
          ([\Gamma]A) \arr ([\Gamma]B) &
      \\{}
      [\Gamma](\alltype{\alpha} A)
         & = & 
         \alltype{\alpha} [\Gamma]A &
      \\[2pt]
      [\Gamma](A, B) 
         & = &
         ([\Gamma]A, [\Gamma]B)
      \\[2pt]
      [\Gamma]\Unionname[A_1,\dots, A_n]
         & = &
         \Unionname[[\Gamma]A_1,\dots, [\Gamma]A_n]
  \end{array}
  \]
  \captionsetup{justification=centering}\caption{Applying a context, as a substitution, to a type}
  \FLabel{fig:ctx_subst}
  \end{minipage}
\end{figure}
