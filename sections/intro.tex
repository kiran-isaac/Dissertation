\chapter{Introduction}
\label{chap:context}

In this dissertation I present SFL-explorer: a tool to demonstrate how functional programming languages are evaluated, allowing users to gain a valuable intuition of these languages.

SFL-explorer takes the form of a functional language (\ac{SFL}), packaged with two interfaces that allows users to observe the process of evaluation of a term as a series of step by step or multi-step reductions, and control the order that sub-terms are evaluated. These interfaces are a \ac{CLI} and a web application. The ultimate goal of this project was to make a tool that makes learning and teaching the basics of functional programming easier. There are two groups of people the project is designed to be of interest to:
\begin{itemize}
    \item Those involved in learning functional languages. These could be students of a university course, or anyone interested in the topic. 
    \item Those involved in teaching functional languages, as part of a university course or otherwise.
\end{itemize}

The language itself is not meant to be the main interest for the users of this system. It is designed to be fairly generic, with syntax and semantics similar to popular functional languages, so that users can take their understanding from using SFL-explorer and apply it to these languages. 

