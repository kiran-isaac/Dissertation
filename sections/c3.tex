\chapter{Cycle 3 --- Improving the UI/UX}
The main focus of this cycle is to implement the next UI iteration, as well as to improve the language. 

\section{Requirements Analysis}
The motivations for this cycle come mainly from the advanced focus group, however requirements from the \hyperref[sec:c1_autoethnography]{autoethnographic phase} of the project, as well as the \hyperref[eval:c1]{proof of concept client meeting} continue to be relevant. 

The advanced focus group was generally very positive about the language, but they had many thought about the Proof of Concept UI they were presented with. During cycle 2, I created a Figma prototype for the next UI (see \ref{c2:next_ui}). This prototype was presented to the advanced focus group, who much preferred it. Many of their thoughts about the Proof of Concept UI were things that were already addressed with the new design. 

With the advanced focus group, following our discussion of the existing system with the proof of concept UI, some tweaks to this design 


\section{The Intermediate Focus Group: Evaluation and Next Steps}
\label{eval:IFG}
Evaluation:
- They liked explicit match: they liked it more than haskell for learning about how pattern matching works
- Really Really needed light mode
- Horizontal overflow bug