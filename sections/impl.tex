\chapter{Implementation}
\label{chap:execution}


\section{OLD PARSER BIT}
\subsubsection{Parsing Match Statements}
An example of using a match statement follows:
\begin{verbatim}
lengthIsAtLeast2 list = match list {
  | Cons x (Cons y xs) -> true
  | _ => false
}
\end{verbatim}

The algorithm used for parsing match statements is:
\begin{itemize}
    \item Consume the "match" keyword.
    \item Parse the expression matched over
    \item Consume an open brace
    \item While the next token isn't a close brace: \begin{itemize}
        \item Parse a pattern (\ref{impl:parsing_patterns}).
        \item Consume a right arrow
        \item Parse an expression
    \end{itemize}
    \item Consume a close brace
\end{itemize}
Following this, a match node is created, where the \verb|children| vector is set appropriately with the pattern and expressions.

\paragraph{Patterns}
\label{impl:parsing_patterns}
A pattern must be a value \ref{design:values}; a pattern must not contain anything that can be reduced. It would be nonsensical to have a situation where we had a pattern not in normal form such as \verb|1 + 1| and the expression to be matched was \verb|2|. 

To parse a pattern, we may use the same techniques as parsing an expression, with a few differences:
\begin{itemize}
    \item Disallowing abstractions
    \item Identifiers must be either:
    \begin{itemize}
        \item Unbound lowercase variables
        \item Underscore (\verb|_|) representing a wildcard pattern
        \item A bound uppercase variable (a constructor)
    \end{itemize}
\end{itemize}



\subsection{Module Parsing}

\section{Type Checking}

\section{Identifying Redexes}


\section{The CLI}

